\documentclass{article}
\usepackage{graphicx}
\usepackage{amssymb,amsmath}

\author{Stefano Duca \and Alexandros Rigos}
\title{Contracts in an Evolutionary Prisoner's Dilemma Game: A Report}

\begin{document}

\maketitle

\section{The main idea}
Here we are playing what we call a contract game in a prisoner's dilemma on a spatial lattice.

The idea is that a player can offer a part of what he earns ($\Delta$) to the other player to make it convienent to cooperate, conditional on the event of both players cooperating. Of course he might be tempted to defect but, due to the preference of errors, maybe the cooperation will be stable.

Here is the payoff matrix of the game:
%\begin{table}
%\centering
\begin{center}
\begin{tabular}{|c|c|c|c|}
\hline 
• & C & D & Delta \\ 
\hline 
C & r,r & -a,1 & r+$\Delta$, r-$\Delta$ \\ 
\hline 
D & 1,-a & 0,0 & 1, -a \\ 
\hline 
Delta & r- $\Delta$, r+$\Delta$ & -a, 1 & r,r, \\ 
\hline 
\end{tabular} 
\end{center}
%\end{table}

The player play a logit best response with parameter delta, responding to the strategy array of the previous round.

We implement the idea in several different ways:

Logit best response

Logit best response with 4 strategies and directional transitions 

Imitation

Imitation with delta conditional on last round's decisions.

\section{Future}

\subsection{Best responders with imitators}

\section{other ideas}
group scoring with stars and intermediate rounds of interaction
%\section{Notes on the code}
%\begin{enumerate}
%\item  So far we consider 3 types of lattice:
%0 stands for squared lattice with Von Neumann n.n. ;
%1 stands for squared lattice with Moore n.n.;
%2 stands for ring lattice.

%\item 0 is C, 1 is D and 2 is Delta
%\end{enumerate}

\end{document}
