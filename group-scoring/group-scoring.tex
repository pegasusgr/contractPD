\documentclass[12pt,a4paper]{article}

\title{Group Scoring: Some Ideas}


\author{Stefano Duca \and Alexandros Rigos}

\begin{document}
\maketitle
\section{Idea for the Model}

We want to work on group scoring but based on the original paper by Nowak (1998): ``Evolution of indirect reciprocity by image scoring.''

The difference between our idea and  Nax, Perc, Szolnoki and Helbing (2015), is that we want to study the system under replicator dynamics.

In each generation $N$ agents are born. They all have a score $\star_i^t$ starting with the value $\star_i^0=0$ for all agents $i$ and a hard-wired strategy: either Cooperate (C) or Defect (D). They are assortatively drawn into groups of size $s$ based on their score. Ties are broken at random. Let $g_i$ denote the set of agents in agent $i$'s group except for agent $i$. Each agent $i$ executes his/her strategy in his/her respective group and receives cumulative payoff according to a linear public goods game played within the group (see Nax et al.). He/she also learns how many agents, other than him/herself, in his/her group played C; indicate this number by $l_i$.

Then, each agent assigns $l_i$ stars to his/her fellow group members.\footnote{Note that other rules are possible and will be explored.} So, each agent $j$ is assigned $\star_j^t=\star_j^{t-1}+\sum_{i\in g_j} l_i$ stars by his/her group-mates. 

Agents are then pooled together and assortatively re-matched based on their updated score. They then play again, receive stars and so on. The process is repeated for $k$ times.\footnote{One could also assume that $k$ follows some probability distribution which can be explored later.} 

After that, each agent produces a number of offspring proportional to his/her fitness and expires. Offspring adopt their parents' strategies. The population is then renormalized to $N$, and star scores and payoffs are reset to zero. This procedure is being repeated for many generations.
\section{Possible Changes}
\begin{itemize}
	\item One could let the decision on whether to award stars to be replicable or to follow some reinforcement learning heuristic.
	\item Agents could be let to condition their decision to award stars based on the star rating of their team members.
	\item Awarding stars can be costly.
	\item Social preferences. Agents may make a ``rational'' decision on whether they want to award stars or not based on social preferences. The idea is that giving stars helps society and therefore socially-minded individuals would prefer to do so.
\end{itemize}

\section{What we expect}
$\Rightarrow$ COOPERATION! 

Why? Because of the haystack model. It shares some similar concepts. 

\end{document}
